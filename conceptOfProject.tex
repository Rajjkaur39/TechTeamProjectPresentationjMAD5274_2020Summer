\section *{\LARGE 1.The Vision and Concept our Project}

The name of our application is “CoronaSafe”. It’s an android application that gives people knowledge about the probability of them testing positive for the Corona virus. Nowadays, it is essential for the public to have an application that is easy to use and updates them on latest news of Corona virus. Thus, this application provides users a wide range of functionalities that help users understand the COVID-19 pandemic situation.
\\
These functionalities enable a user to easily create an account by using Google and Facebook. The public is also allowed to buy medical equipment through the app such as masks and gloves. 

\subsection{FEATURES OF OUR APP}
In this section, you must explain what features you will build for your app.
\begin{enumerate}
    \item SCREENING TOOL: user can test himself whether the person is suffering from corona virus through some series of ques(on and in the end it show the result depends on the answer selected by user.
    \item GLOBAL STAT: Allows the public to know the total global data. For example, the total cases, total death, Active cases through figures, charts and data
    \item LATEST FEED: It shows latest news and updates regarding health and medical stats of the Corona virus.
    \item MEDICAL SHOP: Provides the user the ability to purchase medical products (medicine, mask, sani-tizer
    \item ADD CARD: here user can add payment method to buy medical resources
    \item ADD ADDRESS: user can add delivery address
    \item ORDER HISTORY: this show the product list that user buy through this app
    \item LOGOUT: this button used to logout from application and also clear the session of user identification of the tasks to be accomplished
\end{enumerate}
\subsection{OBJECTIVES OF OUR APP:}
\begin{enumerate}
    \item Person can login and register.
    \item Firebase authentication use for the app login
    \item latest feeds section on COVID-19
    \item 3D world which show corona effect world wide
    \item google and Facebook sign in.
    \item user can test himself or for someone else wether the person is suffering from coronavirus or not with the help of COVID-19 Screening Tool
    \item Direct emergency call functionality.
    \item user can buy medicine , mask and sensitizer from app for revenue generation.
\end{enumerate}
\subsection{TECHNOLOGY TO BE USED:}
Platform: Android\\
Reasons of using Native Application
\begin{enumerate}
\item Very fast
\item Built to run on specific platform
\item Distributed in Play stores.
\item Interactive and intuitive
\item Interact with device utilities.
\begin{itemize}
    \item Language: JAVA
    \item Database: Firebase 
\end{itemize}
\end{enumerate}
\subsection{PHASES OF OUR APPLICATION}
\begin{enumerate}
    \item SPECIFICATION PHASE\\
In the specification phase, the features that will be there application are:
\begin{itemize}
    \item Every user can create separate account.
    \item Firebase authentication use for Secure login
    \item Easy login by google and Facebook sign in.
    \item User can test himself whether the person is suffering from corona virus or not with the help of COVID-19 Screening Tool
    \item 3D world which show corona effect worldwide
    \item latest feeds section on COVID-19.
    \item user can buy medicine, mask and sensitizer from app for revenue generation.
\end{itemize}
\item REALIZATION PHASE\\
While creating this application, we have implemented all the features mentioned above. However, as we were developing this application, we noticed we have 2 more functionalities that we did not include in the specification phase. These include the payment method and the delivery address to complete the medical shopping section. We implemented this functionality while completing this project to ensure the application ran smoothly.
\end{enumerate}

\subsection{The context of study}
In the third semester of our program we have a course of developing a Capstone project. We have to choose a topic which can solve a real-life problem and develop an application out of it. So, the topic we have chosen is the recent pandemic of the Corona Virus
\begin{enumerate}
    \item MOTIVATION TO CHOOSE THIS TOPIC: We are currently experiencing a pandemic that is negatively affecting the health of many people across the world. There are many countries who do not have access to the correct information, resources, and news about the virus. Due to this, the public is not aware about the severity of the situation. Thus, the focus for the application is to provide the public with the resources they need to tackle the Corona virus situation and to inform them about the latest updates. This will ensure the world is on the same page about the pandemic.
    \item PROBLEM SOLVED BY OUR APP: This app will provide a way for all the people in the world to stay updated on the Corona virus pandemic, provide medical resources to the public in an easy functional way and test themselves for the virus. This app will help medical officials to tackle the virus in an effective way.
    \item DEFINING AND ANALYZING THE PROBLEM
    \begin{itemize}
    \item Animation: we used Complex animation for the first time in our app. Thus, it took a while
    \item New to fire base fire store: we used different fire base services before not fire store so that’s another difficulty we faced
    \item Difficulty in finding free API's: we need an API for words in our application, but we are finding it difficult to find a free API.
\end{itemize}
\end{enumerate}
